
\chapter{Conclusion}

\section{Overview}
This final chapter will collate all previously discussed results and assess what this research has provided to the academic community within \gls{ux} research. The projects aims and objectives will then be assessed. Lastly the projects limitations and any suggestions for future work shall be discussed.

\section{Project Conclusions}
The main aim of this project was to assess the influence that previous user knowledge has upon the Concurrent Think-Aloud methodology. Therefore a Concurrent Think-Aloud study using PC gaming service Steam made by the Valve Corporation was conducted. A total of thirty-two participants were involved split into two equal groups, Novice and Experienced. These participants were asked to attempt six tasks within the Steam client, which encompassed many of the common functions used in the program, such as searching the Store for purchasable games or changing the \gls{ui} for a more console like experience. When each of this tasks were attempted, several data metrics were collected. This metrics included Task Duration, Task Completion, Task Mouse Interactions, Reception to the Think-Aloud Protocol, Usability Assessment of the Steam service, and crucially the frequency and types of issues or errors encountered in the test process. Each participant experienced similar conditions, regarding environment and the equipment used in the testing. This data was then organised using Microsoft Excel and analysed using SPSS.

The results revealed that Novice and Experienced participants do have show significant differences in regard to some Tasks, especially those which are more complex, which in this study was Task C and D and therefore Experienced participants are more efficient at using services, but in this study found less issues overall. This is likely because Steam is a highly developed software which has been on the market since 2004. However several similarities were also present between the two types of participant. Firstly, some of the same issues were detected by both participants, which some studies have suggested is not the case. Additionally both types responded similarly when asked about Thinking-aloud. Lastly, Novice and Experienced participants react similarly to certain tasks, as has been seen in Task A, where all participants where able to complete the Task successfully. 

\section{Project Objectives and Aim Evaluation}

This section will now evaluate the success project via an assessment of the Objectives and overall project Aim. The first objective was \textit{to review and discuss relevant literature around issues of usability and the Think-Aloud Methodology}. This objective was met in Chapter 2 - Literature Review, as discussion surrounding the field of usability testing and more specific works regarding the use and assessment of previous user experience in past studies is present. The second objective was \textit{to arrange and execute a Think-Aloud usability study with the expectation of 40 participants split across two categories of Novice and Experienced users, using the game distribution service Steam}. This has been achieved, as evidenced by the multiple chapters, especially chapters 4 and 5 which present the results of this study. The next objective was \textit{to provide critical analysis of collected data from Novice and Experienced participants and draw conclusions on any significant outcomes}, this objective has been completed as evidenced by Chapter 6 - Discussion, which deduces several findings based on the analysed data sets. Lastly the final objective was \textit{to provide suitable feedback based on the outcome of the study to the Valve corporation to improve the Steam service}, although this objective is secondary, it still has been met in Chapter 7 - Recommendations for the Steam service,  which discusses some theoretical options for improvements of the Steam software. Therefore all Objectives can be considered met, and thus the Aim of the project also.

\section{Limitations}
During the process of conducting this project I have encountered several limitations and problems that should be addressed, should this projects work be expanded upon in the future. The largest set of limitations revolves around the Demographics of the Experienced participants compared against the make-up of the Novice sample. The majority of participants in the Experienced category were aged between 18-30 and were male, with only three female participants out of sixteen. Conversely the Novice category had a more balanced set of demographics with an equal amount of male and female participants, and a greater range of participant ages. A general flaw from this project also included the fact that the intention of having forty participants split into twenty Novices and twenty Experienced participants was not realised, with only thirty-two participants (16 Novices, 16 Experienced) thus having 8 missing participants that were expected.  The second aspect that could be expanded upon is the Task list itself. The six tasks tested the most commonly used functions of the Steam interface. However other aspects were not examined. For instance, Steam offers a service were users can purchase and trade in game items usually cosmetic in nature, known as the Steam Marketplace, a task could be based upon this, however if too many Tasks are present then a participant may become tired and behave differently than when they first started the test. Thirdly, this project also uncovered issues encountered with the terminology used within the Tasks, especially when they were attempted by the Novice participants, this was seen in the word verification Task D, and Big Picture Mode in task E.

\section{Suggestion for Further Work}
With the conclusion of this project, there are various aspects that could be expanded upon in future works. Firstly, as mentioned above this sample had the expectation of forty participants, however future studies could have a larger pool of participants compared to this, perhaps in the hundreds. However this would be very time consuming and potentially not beneficial to researchers or specific tested platforms. Additionally any future work could involve more specialised forms of knowledge experience groups such as Novice, Experienced, Expert. (Faulkner and Wick 2005). A greater and more diverse set of participants could have led to a greater amount of issues detected, as compared with this study results. Regarding the data collection phrase of the research, OBS and WhatPulse were used to record a visual record and count the numbers of Mouse interactions per tasks. However one such feature that could have been used in this study was the use of Heat maps, these heat maps could have been used on a Task by Task basis as well as overall usage of the Steam software. These heat maps would have shown the distribution of the areas of interactions, which adds another layer of metrics to this study, as well as pinpointing visually the most commonly used areas of Steam, and what areas are less frequently interacted with.
As already discussed in section 8.2, Steam is a mature piece of software, which has undergone a large amount of changes over its lifetime and as such issues encountered within the program are likely fewer than a newer piece of software or even an incomplete program or website. Additionally this study assessed the impact of previous user knowledge on the Concurrent-Think Aloud methodology, however as highlighted by Chapter 2 other Think-aloud methods exist and as such another study could be made involving these methods, especially Retrospective Think-aloud testing, during which I suspect that Novice users would react differently to that of Concurrent.

\section{Summary}
This research has tried to provide detailed assessment of the impact that Novice and Experienced participants have upon Concurrent usability testing, alongside this studies limitations and suggestions for further work in the evaluation of previous user experience in the testing of applications. 